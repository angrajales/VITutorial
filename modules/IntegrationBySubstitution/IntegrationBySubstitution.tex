\documentclass[14pt,dvipsnames, aspectratio=169]{beamer}

\usetheme{Montpellier}
\usecolortheme{beaver}

\usepackage{physics}
\usepackage{amsmath, amssymb, ../../vimacros, hyperref, tikz}
\usetikzlibrary{positioning, fit, bayesnet, shapes.misc, patterns}
\usepackage[round]{natbib}
\usepackage{mathalfa}
\usepackage{cancel}
\usepackage{verbatim}

\beamertemplatenavigationsymbolsempty

\hypersetup{breaklinks=true, colorlinks=true, linkcolor=blue, urlcolor=blue, citecolor=blue}

\newcommand{\balert}[1]{\textcolor{blue}{#1}}
\newcommand{\galert}[1]{\textcolor{PineGreen}{#1}}

\makeatletter
\newenvironment{noheadline}{
    \setbeamertemplate{headline}{}
    \addtobeamertemplate{frametitle}{\vspace*{-0.9\baselineskip}}{}
}{}
\makeatother

\title{Integration By Substitution}
\author{Philip Schulz and Wilker Aziz\\
\url{https://github.com/vitutorial/VITutorial}}
\date{}

\setbeamertemplate{footline}[frame number]

\begin{document}

\begin{frame}
\maketitle
\end{frame}

\begin{frame}{Multivariate calculus recap}

Let $x \in \mathbb R^K$ and let $\mathcal T: \mathbb R^K \to \mathbb R^K$ be differentiable and invertible
\begin{itemize}
	\item $y = \mathcal T(x) $
	\item $x = \inv{\mathcal T}(y)$
\end{itemize}

\end{frame}

\begin{frame}{Jacobian}

	The Jacobian matrix $\jac{\mathcal T}{x} $ of  $\mathcal T$ \\
	~assessed at $x$ is the matrix of partial derivatives
	\begin{equation*}
		J_{ij} = \pdv{y_i}{x_j} = \pdv{\mathcal{T}(x)_{i}}{x_{j}}
	\end{equation*} 
	
	\pause
	Inverse function theorem
	\begin{equation*}
		\jac{\inv{\mathcal T}}{y} = \left( \jac{\mathcal T}{x} \right)^{-1}
	\end{equation*}
	
\end{frame}

\begin{frame}{Differential (or inifinitesimal)}

	The {\bf differential} $\dd x$ of $x$ refers to an \emph{infinitely small} change in $x$\\ \pause
	\vspace{10pt}

	We can relate the differential $\dd y$ of $y = \mathcal T(x)$ to $\dd x$ \pause
	\begin{block}{Scalar Case}
		\begin{equation*}
			\dd y \dv{x}{x} = \pause \alert{\dv{y}{x}} \dd x = \alert{\dv{\mathcal{T}(x)}{x}} \dd x \pause \implies \dd x = \dv{x}{y} \dd y
		\end{equation*}
		\pause
		\begin{itemize}
		\item \alert{$\dv*{y}{x}$} scales the differential $ \dd x $ to match it to $ \dd y $
		\item if $ \alert{\dv*{y}{x}} > 1 $, $ \mathcal{T} $ expands the area around $ x $ locally
		\end{itemize}
	\end{block}

\end{frame}

\begin{frame}{Differential (or inifinitesimal)}

	The {\bf differential} $\dd x$ of $x$ refers to an \emph{infinitely small} change in $x$\\ \pause
	\vspace{10pt}

	We can relate the differential $\dd y$ of $y = \mathcal T(x)$ to $\dd x$ \pause
	\begin{block}{Multivariate Case}
		\begin{equation*}
        		\begin{aligned}
			     \dd y = \alert{\djac{\mathcal T}{x}} \dd x % = \alert{\abs{\pdv{x}T(x)}} \dd x % in some texts people will find this notation
		     \end{aligned}
	     \end{equation*}
		the absolute value absorbs the orientation 
	\end{block}

\end{frame}



\begin{frame}{Integration by substitution}	
	We can integrate a function $g(x)$ \\
	~ by substituting $x = \inv{ \mathcal T}(y)$
	\begin{equation*}
	\begin{aligned}
		\int g(\balert{x}) \alert{\dd x} \pause &= \int g(\underbrace{\balert{\inv{\mathcal T}(y)}}_{x}) \underbrace{\alert{\djac{\inv{\mathcal T}}{y} \dd y}}_{\dd x} \\ \pause
	\end{aligned}
	\end{equation*}
	
	\vspace{-10pt}
	and similarly for a function $h(y)$
	\begin{equation*}
	\begin{aligned}
		\int h(\balert{y}) \alert{\dd y} \pause &= \int h(\balert{\mathcal T(x)}) \alert{\djac{\mathcal T}{x} \dd x}
	\end{aligned}
	\end{equation*} 

\end{frame}

\begin{frame}{Change of density}

Let $X$ take on values in $\mathbb R^K$ with density $p_X(x)$\\ \pause
~ and recall that $y = \mathcal T(x)$ and $x = \inv{\mathcal T}(y)$\\ \pause

~

Then $\mathcal T$ induces a density $p_Y(y)$ expressed as
\begin{equation*}
p_Y(y) = p_X(\inv{\mathcal T}(y)) \djac{\inv{\mathcal T}}{y}
\end{equation*} \pause
and then it follows that
\begin{equation*}
p_X(x) = p_Y(\mathcal T(x)) \djac{\mathcal T}{x}
\end{equation*}

	
\end{frame}

\end{document}